\documentclass[10pt]{article}

%%--------------------------------------------------------------------------- 
%% LaTeX Configuration
%% Most of this is lifted from NIH_Center.tex by Eric Bohm
%%--------------------------------------------------------------------------- 
%\usepackage{graphicx}
\usepackage{enumerate}
\usepackage{setspace}
\usepackage{times}
\usepackage{psfig}
\usepackage{epsfig}
\usepackage{epsf}
\usepackage{subfigure}
\usepackage{calc}
\usepackage{alltt}
\usepackage{amsfonts}
\usepackage{amsbsy}
\usepackage{amsmath}
\usepackage{color}
\usepackage{cite}
\usepackage{listings}
\usepackage{amsmath}
\usepackage{amsthm}
\usepackage{newalg}

\newtheorem{theorem}{Theorem}
\newtheorem{proposition}{Proposition}
\newtheorem{corollary}{Corollary}
\newtheorem{conjecture}{Conjecture}
\newtheorem{lemma}{Lemma}
\newtheorem{definition}{Definition}

%%% in-band figures
\renewcommand{\textfraction}{0.0}
\renewcommand{\topfraction}{1.0}
\renewcommand{\bottomfraction}{0}
\renewcommand{\floatpagefraction}{.99}

\oddsidemargin  -0.2 in
\evensidemargin -0.2 in
\textheight      9.0 in
\textwidth       6.8 in
\topmargin      -0.3 in

\parskip 3pt plus 1pt
\baselineskip 13pt 
\itemsep 0em
\topsep 0em

\def\Comment#1{\textbf{\textsl{$\langle\!\langle$#1$\rangle\!\rangle$}}}

%%--------------------------------------------------------------------------- 
%% Main Document
%%--------------------------------------------------------------------------- 

\title{\bfseries{SSCA2 Kernel 4 Algorithm Description}} 
\author{Robert L. Bocchino Jr.}
\date{September 2007}

\bibliographystyle{plain}

\begin{document}

\begin{sloppypar}

\maketitle

This document describes my implementation of SSCA2 Kernel 4 (graph
clustering), based on the reference implementation provided in version
1.0 of the benchmark.  I describe both the explicit locks version and
the STM version.  Please note that \emph{both versions are
speculative}.  However, the locks version encodes the speculation
``manually'' at the application level, while the STM version leverages
the speculative mechanism built into the STM runtime.  Similarly, the
locks version uses explicit locks, while the STM version uses the
locks built into the STM.  (In theory, the STM implementation could
also be lock-free, but mine isn't.)

Both versions of the algorithm use the following processor-local,
function-global data structures, one on each processor $p$:
\begin{itemize}
\item $V_p$:  The set of vertices local to $p$.  The vertices adjacent
  to each $v \in V_p$ are stored in an adjacency list representation,
  also on $p$.  The adjacency list is modified when a vertex in the
  list is added to a cluster (``claimed'') so that subsequent users of
  the list will no longer ``see'' the link to the cluster (this
  effectively removes each vertex in the cluster from the graph).
\item \emph{candidateSet}:  The set of candidate vertices for
  inclusion in the cluster we are currently computing
\item \emph{adjSet}:  The set of vertices adjacent to the candidate
  set
\item \emph{adjCache}: A map that locally stores the adjacency list of
  each vertex that we have locked.  The set of keys in the map also
  records which vertices we have locked.
\item \emph{cluster}: The actual current cluster, a subset of
\emph{candidateSet}
\item The \emph{output}, consisting of a set of clusters.
\end{itemize}
There is also an array (not explicitly shown) that associates a
\emph{state} with each vertex.  

\section{Explicit Locks Implementation}

In the explicit locks version, the possible vertex states are CLAIMED,
which means that some processor has permanently added this vertex to a
cluster; AVAILABLE, which means that the vertex is available for any
processor to add to a cluster; and LOCKED, which means that some
processor has provisionally added the vertex to its cluster, but has
not yet committed the cluster.  The LOCKED state also encodes which
processor owns the lock; this information is used for a simple
priority-based deadlock avoidance protocol.  The vertex state array is
distributed across the processors in the same way as the vertices and
is visible to all processors.

\subsection{Main Algorithm}

\emph{Pseudocode}.  Figure~\ref{figure:main} shows the pseudocode for
the main algorithm.  Each processor iterates over its local vertices,
as shown in line 4.  If the vertex is claimed, then the processor
moves on.  If the vertex is locked, then the processor keeps trying
that vertex until it is claimed or released.  Once a processor locks a
local vertex, it attempts to build a \emph{candidate set} starting
with that vertex, as shown in line 7.  This procedure may fail,
because it involves locking more vertices.  If it fails, then the
processor unlocks everything and tries again.  Otherwise, it uses the
candidate set to compute a cluster, updates the vertex data (state and
adjacency lists) possibly residing on other processors, and adds the
cluster to its local output, as shown in lines 8--10.

\emph{Supporting functions}.  The supporting functions are
\emph{lock}, \emph{compute\_candidate\_set}, \emph{update\_graph},
\emph{abort}, and \emph{compute\_cluster}.  Section~\ref{section:lock}
explains the function \emph{lock}, section~\ref{section:compute_cset}
explains the function \emph{compute\_candidate\_set}, and
section~\ref{section:update} explains the function \emph{update\_graph}.  The
function \emph{abort} sets the state of each vertex in \emph{adjCache}
to AVAILABLE, using the same per-processor aggregation pattern as
\emph{lock}.  It is very straightforward, because it cannot fail
(unlike \emph{lock}).  I omit its pseudocode.  The function
\emph{compute\_cluster} is a local computation: if \emph{candidateSet}
is sufficiently large (bigger than a prescribed minimum) it sets the
cluster to be the first $n$ elements of \emph{candidateSet} such that
the number of adjacencies between those elements and the whole set is
minimized.  If \emph{candidateSet} is not sufficiently large, it just
uses the whole candidate set.  I omit the pseudocode for this
function.  For further details, see the SSCA2 specification.

\begin{center}
\begin{figure}
\begin{algorithm}{SSCA2K4}{\text{Graph $G = (V,E)$}}
\text{Distribute $V$ evenly over the processors $P$ into sets $V_p$}
\\
\begin{FOR}{\text{each processor $p$ in parallel}}
\begin{WHILE}{V_p \neq \emptyset}
\text{\emph{adjCache}} \leftarrow \emptyset \\
v \leftarrow \text{remove one element of $V_p$} \\
\text{\textbf{if} $v$ is claimed \textbf{then} continue} \\
\text{\textbf{if} \CALL{lock}$(v) =$ FAIL \textbf{then} add $v$ to
  $V_p$; continue} \\
\text{\textbf{if}} \CALL{compute\_candidate\_set}(v) = \text{FAIL \textbf{then}}
\CALL{abort}(); \text{add $v$ to $V_p$}; \text{continue} \\
\text{\emph{cluster}} \leftarrow \CALL{compute\_cluster}() \\
\CALL{update\_graph}() \\
\text {add \emph{cluster} to \emph{output}}
\end{WHILE}
\end{FOR} \\
\text{Merge output of all processors}
\end{algorithm}
\caption{Pseduocode for SSCA2 Kernel 4, locks implementation}
\label{figure:main}
\end{figure}
\end{center}

\subsection{The \emph{lock} Function}
\label{section:lock}

Figure~\ref{figure:lock} shows the pseudocode for the function
\emph{lock}.  We lock only the vertices not present in
\emph{adjCache}, because any vertex in \emph{adjCache} has already
been locked.  We sort the vertices to lock by processor.  For each
processor, we send out all vertices in one message, and we perform the
lock operation locally on the destination processor.  We get back an
array of results.  We then iterate through the array, checking the
status of each vertex.

If the status of a vertex is CLAIMED, then we will never be able to
acquire this vertex (some other processor has claimed it), so we give
up on this cluster and return FAIL.  If the status is AVAILABLE, then
the local lock operation has locked the vertex for us and returned its
adjacency list; we put the list into \emph{adjCache} for later use.
If the status is neither CLAIMED nor AVAILABLE, then some other
processor has locked the vertex, and the status tells us which
processor that is.  We compare the status with our own processor ID to
see our relative priority to the lockholder.  If we are lower
priority, we abort.  If we are higher priority, we put the vertex back
into the vertex set and try to get it on the next pass, knowing that
the lockholder will eventually succeed or abort (it will never wait
for a higher priority processor, and in particular will never be
connected to us by a chain of wait-for edges).  We keep iterating over
the vertex set until it is empty.

\begin{center}
\begin{figure}
\begin{algorithm}{lock}{\text{vertices $V$}}
\begin{WHILE}{V \neq \emptyset}
\text{sort $V \backslash \CALL{keys}(\mbox{\emph{adjCache}})$ by processor into sets $V_p$}\\
V \leftarrow \emptyset \\
\begin{FOR}{\text{each processor $p$ such that $V_p$ is nonempty}}
\text{\emph{results} $\leftarrow$ \textbf{on} $p$ \CALL{lock\_local}$(V_p)$} \\
\begin{FOR}{\text{each $r \in$ \emph{results}}}
\text{\textbf{if} $r$.\emph{status} $=$ CLAIMED \textbf{then return} FAIL} \\
\text{\textbf{else if} $r$.\emph{status} $=$ AVAILABLE \textbf{then}
\emph{adjCache}$[v] \leftarrow r.$\emph{adjList}} \\
\text{\textbf{else if} lockholder is higher priority \textbf{then return} FAIL} \\
\text{\textbf{else} add $v$ to $V$}
\end{FOR}
\end{FOR}
\end{WHILE}\\
\RETURN{\text{SUCCEED}}
\end{algorithm}
\caption{Pseudocode for the \emph{lock} function}
\label{figure:lock}
\end{figure}
\end{center}

\subsection{The \emph{compute\_candidate\_set} Function}
\label{section:compute_cset}

Figure~\ref{figure:compute_cset} shows the pseudocode for the function
\emph{compute\_candidate\_set}.  This function computes a ``candidate
set'' (a superset of a cluster) starting with a single ``seed'' vertex
$v$.  We maintain an \emph{adjSet}, which consists of all vertices
adjacent to the candidate set.  Initially, the candidate set contains
$v$, and \emph{adjSet} contains the adjacencies of $v$, which we look
up in the \emph{adjCache}.  We try to lock the \emph{adjSet} and abort
if any of the locks fails.

We iteratively add points to the candidate set.  In each iteration, we
do the following:
\begin{enumerate}
\item Select a next candidate from the current \emph{adjSet}.  The
selection uses a simple heuristic to look for a vertex that is most
tightly connected to the cluster.  I omit the pseudocode, but more
detail can be found in the SSCA2 specification.
\item Try to lock the adjacencies of $v$ that are not already in our
  candidate set.  One of these vertices will become our next
  candidate.  If this operation fails, abort.
\item Add $v$ to the candidate set and remove it from \emph{adjSet}.
  Add the new adjacencies of $v$ (i.e., adjacencies that aren't
  already in the candidate set) to the candidate set.
\end{enumerate}
We continue iterating until (1) there are no more vertices adjacent to
the candidate set or (2) the candidate set is sufficiently large.

\begin{center}
\begin{figure}
\begin{algorithm}{compute\_candidate\_set}{\text{vertex $v$}}
\text{\emph{candidateSet} $\leftarrow$ $\{v\}$} \\
\text{\emph{adjSet} $\leftarrow$ \emph{adjCache}$[v]$} \\
\text{\textbf{if} \CALL{lock}$($\emph{adjSet}$)$ = FAIL \textbf{then return} FAIL} \\
\begin{WHILE}{\text{\emph{adjSet}} \neq \emptyset \text{\textbf{and}} 
|\text{\emph{candidateSet}}| < \text{MAX\_CLUSTER\_SIZE}}
v \leftarrow \CALL{next\_candidate}(\mbox{\emph{adjSet}}) \\
\text{\textbf{if} \CALL{lock}$(\mbox{\emph{adjCache}}[v] \backslash \mbox{\emph{candidateSet}}) =$ FAIL 
\textbf{then return} FAIL} \\
\text{add $v$ to candidateSet} \\
\text{remove $v$ from \emph{adjSet}} \\
\text{add \emph{adjCache}$[v] \backslash \text{\emph{candidateSet}}$ to \emph{adjSet}}
\end{WHILE} \\
\RETURN{\text{SUCCEED}}
\end{algorithm}
\caption{Pseudocode for the \emph{compute\_candidate\_set} function}
\label{figure:compute_cset}
\end{figure}
\end{center}

\subsection{The \emph{update\_graph} Function}
\label{section:update}

Figure~\ref{figure:update} shows the pseudocode for the function
\emph{update\_graph}.  This function updates the state and
adjacency information for three kinds of vertices: (1) vertices in the
cluster; (2) vertices in the candidate set but not in the cluster; and
(3) vertices adjacent to the candidate set.  We sort each
set of vertices by processor.  Then we iterate over the processors
that have any vertices to update.  For each processor $p$, we compute the
updated adjacencies of the vertices in set (3), removing any links to
the cluster.  This procedure effectively removes the cluster vertices
from the graph.

We then call the function \emph{update\_graph\_local} on processor $p$.  This
function accepts all the update information for $p$ in one
communication.  It then iterates over all the vertices that need
updating, and performs the appropriate update.  For vertices in set
(1), this means setting the state to CLAIMED; for set (2), it means
setting the state to AVAILABLE; and for set (3), it means updating the
adjacencies and setting the state to AVAILABLE.

\begin{center}
\begin{figure}
\begin{algorithm}{update\_graph}{\text{\emph{cluster}, \emph{adjSet}, \emph{adjCache}}}
\text{sort \emph{cluster} by processor into sets $C_p$} \\
\text{sort \emph{candidateSet} $\backslash$ \emph{cluster} by processor into sets $C'_p$} \\
\text{sort \emph{adjSet} by processor into sets $A_p$} \\ 
\begin{FOR}{\text{each $p$ such that $C_p \cup C'_p \cup A_p \neq \emptyset$}}
M \leftarrow \emptyset \\
\begin{FOR}{\text{each $a \in A_p$}}
M[a] \leftarrow \text{updated adjacencies of $a$, disregarding links to cluster}
\end{FOR} \\
\text{\textbf{on} $p$} \CALL{update\_graph\_local}(C_p,C'_p,A_p,M)
\end{FOR}
\end{algorithm}
\caption{Pseudocode for the \emph{update\_graph} function}
\label{figure:update}
\end{figure}
\end{center}

\section{STM Implementation}

The STM version is similar to the explicit locks version, except for the following:
\begin{enumerate}
\item The possible vertex states are CLAIMED and AVAILABLE.  There is
no LOCKED vertex state.
\item Instead of locking vertices and checking locks, the
implementation just reads and writes the states and adjacency lists.
The STM's conflict detection mechanism ensures that the same semantics
as in the locking version will be preserved.
\item There is no explicit abort code.  Instead, the STM abort
  mechanism provides the rollback.
\end{enumerate}

The code for the main algorithm of the STM version is shown in
Figure~\ref{figure:main-stm}.  Note the absence of the locking
function and any abort code.  Note also that we must be careful to
reset local data structures at the beginning of the transaction, so
that the abort works correctly.  In particular, the adjacency cache
must be cleared at the start of every transaction.

The supporting functions are similar to the locks version, but
simpler.  Instead of the \emph{lock} function
(Section~\ref{section:lock}), we use a \emph{cache} function that
pulls the adjacency sets for the requested vertices into the
\emph{adjCache} (Figure~\ref{figure:cache}).  The local function
executed on each remote processor uses \emph{stm\_read} to perform the
reads, so that the transaction can abort in the case of a conflict.
Note that we do not need to check for LOCKED states, because the
transactions guarantee serializability.  Nor do we need to check for
CLAIMED states, because if we ever follow an edge in the graph to a
claimed vertex, a conflict will occur: when the transaction that
claimed the vertex commits, it will cut the link that we followed,
causing a read-write conflict.  We use a
\emph{compute\_candidate\_set} function simlar to the locks version
(Section~\ref{section:update}), except that we call \emph{cache}
instead of \emph{lock} and do not explicitly check for conflicts
(Figure~\ref{figure:compute-cset-stm}).  Finally, the
\emph{update\_graph} function is identical to the function shown in
Figure~\ref{figure:update}.  However, the \emph{update\_graph\_local}
function uses \emph{stm\_write} to update the state and adjacencies,
so that read-write conflicts across transactions will be detected.
(It would probably be sufficient to use \emph{stm\_read} and
\emph{stm\_write} only on the states, and not on the adjacency lists,
because the states ``guard'' the lists.  However, it would be very
difficult for a compiler to prove this fact.)

\begin{center}
\begin{figure}
\begin{algorithm}{SSCA2K4}{\text{Graph $G = (V,E)$}}
\text{Distribute $V$ evenly over the processors $P$ into sets $V_p$}
\\
\begin{FOR}{\text{each processor $p$ in parallel}}
\begin{WHILE}{V_p \neq \emptyset}
\CALL{stm\_start} \\
\text{\emph{adjCache}} \leftarrow \emptyset \\
v \leftarrow \text{remove one element of $V_p$} \\
\text{\textbf{if} $v$ is claimed \textbf{then} continue} \\
\CALL{compute\_candidate\_set}(v) \\
\text{\emph{cluster}} \leftarrow \CALL{compute\_cluster}() \\
\CALL{update\_graph}() \\
\CALL{stm\_commit} \\
\text {add \emph{cluster} to \emph{output}}
\end{WHILE}
\end{FOR} \\
\text{Merge output of all processors}
\end{algorithm}
\caption{Pseduocode for SSCA2 Kernel 4, STM implementation}
\label{figure:main-stm}
\end{figure}
\end{center}

\begin{center}
\begin{figure}
\begin{algorithm}{cache}{\text{vertices $V$}}
\text{sort $V \backslash \CALL{keys}(\mbox{\emph{adjCache}})$ by processor into sets $V_p$}\\
\begin{FOR}{\text{each processor $p$ such that $V_p$ is nonempty}}
\text{\emph{results} $\leftarrow$ \textbf{on} $p$ \CALL{cache\_local}$(V_p)$}
\end{FOR}
\end{algorithm}
\caption{Pseudocode for the \emph{cache} function}
\label{figure:cache}
\end{figure}
\end{center}

\begin{center}
\begin{figure}
\begin{algorithm}{compute\_candidate\_set}{\text{vertex $v$}}
\text{\emph{candidateSet} $\leftarrow$ $\{v\}$} \\
\text{\emph{adjSet} $\leftarrow$ \emph{adjCache}$[v]$} \\
\CALL{cache}$($\emph{adjSet}$)$ \\
\begin{WHILE}{\text{\emph{adjSet}} \neq \emptyset \text{\textbf{and}} 
|\text{\emph{candidateSet}}| < \text{MAX\_CLUSTER\_SIZE}}
v \leftarrow \CALL{next\_candidate}(\mbox{\emph{adjSet}}) \\
\CALL{cache}(\mbox{\emph{adjCache}}[v] \backslash
\mbox{\emph{candidateSet}}) \\
\text{add $v$ to candidateSet} \\
\text{remove $v$ from \emph{adjSet}} \\
\text{add \emph{adjCache}$[v] \backslash \text{\emph{candidateSet}}$ to \emph{adjSet}}
\end{WHILE}
\end{algorithm}
\caption{Pseudocode for the \emph{compute\_candidate\_set} function in
the STM implementation}
\label{figure:compute-cset-stm}
\end{figure}
\end{center}

%\bibliography{references}

\end{sloppypar}

\end{document}
